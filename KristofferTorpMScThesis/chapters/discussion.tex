\documentclass[../main.tex]{subfiles}
\begin{document}

In general we we're able to differentiate movement patterns between various dates using Co-Trajectories. In figure \ref{fig:feb-dist-dend} we see two sets of distance matrices and dendrograms for the February case, the left is based on the L1 distance, and the right on the weighted Jaccard distance. Each row and column represents a date. The checkered pattern indicates a difference between weekends and weekdays. Closer inspection shows that Fridays sits somewhere in-between. Saturdays are more like other Saturdays than they are to Sundays. Generally, the image gets brighter towards the bottom right, possibly due to increased activity towards the Swedish school holiday, the "Sportlov" which starts on the 27th February in the Stockholm region. Over the last two days in February the difference between weekends and weekdays become less pronounced. It is interesting to try and interpret the various values in the difference matrices and they seem to be able to make out some "common-sense" differences.

Likewise, from the dendrograms me can make out various clusters. Both distances cluster weekends and weekdays apart from each other. Further, Saturdays and Sundays are grouped apart. For the weekdays all Fridays are grouped together. Days within the same week are often put together. 

In figure \ref{fig:thu-L1} and \ref{fig:thu-Jaccard} we see the same plots but for a larger sample set, this time for Thursdays during 2022, giving us a view over a whole year instead of a single month. Here we can see the change from Swedish summer and winter times which takes place on the 27th of March and 29th of October respectively. The large bands forming a "plus" in the middle corresponds to the Swedish summer holidays. Some thin strips can be linked to various "red days" which are generally paid vacation days such as the 6th of January and 26th of May.
The dendrogram is slightly more difficult to read but it groups the first three months of the year close together and generally puts days within a month close.

In figure \ref{fig:thu-L1} we see movement profiles for February with L1 distance to the left and Jaccard to the right. Recall that the movement profiles measure changes between consecutive hours, hence the brightest spots indicate hours that are markedly different than the one before. This is why we have spots in the early mornings around 5 in the weekdays and around 9 in the weekends, since people would generally start to travel to work or get out of their houses by those times. The L1 metric displays a more smoothed level of movement throughout the whole day while the Jaccard metric mostly displays critical hours. Likewise in figure \ref{fig:full-mov-pro} we see movement profiles for most Thursdays of 2022 where it is interesting to note especially the summer months which has much weaker contrast in the mornings since people are not beholden to the work commute.

Lastly, in figure \ref{fig:comparison_minhash} the left column shows various MinHash approximations of increasingly larger sample sizes, starting from the top with 64 samples and going down to 128, 256, 512 and 1024. The middle row shows the difference between the MinHash approximation and the true weighted Jaccard matrix (right). For this kind of study the use of the weighted MinHashes turned out to be of little value, since the total number of pairwise comparisons were pretty low, just a few dates throughout the year, but if we instead would like to compare individual trajectories then using the MinHash approximation might become important to keep computations feasible since we can utilize the properties of locality sensitive hashing and quickly find candidate nearest neighbours for clustering.

\end{document}