%%% 0bAbstract/ %%%
\chapter*{Abstract}
\begin{comment}
Sell your idea!
Make the reader want to stay with you!
Single paragraph, 100-200 words

Four parts/sentences
1. What's the problem
2. How did you solve it
3. What are the results
4. Conclusion (what it means for the future)
Make sure the abstract stands on its own!
-No reference tags
- Avoid acronyms

\end{comment}

Road accidents are one of the most important concerns worldwide. 
Studying and analysing them is one way to improve and ensure safety on the roads. 
Traffic accident records for Lithuania over a period of four years have been provided and are analysed in this thesis. 
The data processing is done according to the metadata and data definitions of the European framework for road accident. 
On top of this, a discrete approach to segment the country's road network using Open Street Map data is performed together with a segmentation of the country by municipalities followed by a map-matching algorithm to associate each accident to the nearest road segment of a Pregel-programmable directed property graph. 
The goal is to estimate how the accident count is affected by external variables like the weather conditions or the distance of the accident to its nearest intersection by applying a method of Poisson regression. 
The results show that the factors that most influence the number of accidents are normal weather and light conditions like clear day, daylight and dry surface. 
Given that this dataset does not contain any information on the frequency of vehicles without accidents, this work should be seen as a first step towards a more meaningful Poisson regression that also includes zero counts for the occurrence of no accidents.
